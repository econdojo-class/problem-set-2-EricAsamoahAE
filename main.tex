
%------------%
%  Preamble  %
%------------%

\documentclass[final,11pt]{article}
\usepackage[paperwidth=9.0in, top=1.2in, bottom=1.2in, left=1.2in, right=1.2in]{geometry}
\usepackage{amsmath}
\usepackage{color}
\usepackage{multirow}
\usepackage{setspace}
\usepackage{fancyhdr}
\usepackage{longtable}
\usepackage{array}
\usepackage{booktabs}
\usepackage{mathpazo}
\usepackage{threeparttable}
\usepackage{eurosym}
\usepackage[colorlinks, linkcolor=blue, anchorcolor=blue, citecolor=blue]{hyperref}

\renewcommand{\headrulewidth}{0pt}
\setlength{\arraycolsep}{10pt}
\setlength\headheight{0.5cm}
\setlength\headsep{0.8cm}
\setlength\footskip{1.0cm}
\setlength{\parindent}{0em}
\pagestyle{fancy}
\chead{\textcolor[rgb]{0.5,0.5,0.5}{\sc Spring 2025: ECON 6100}}

%------------%
%  Document  %
%------------%

\begin{document}
\thispagestyle{empty}
\begin{spacing}{1.25}

\textbf{Your Name:Eric Asamoah\hfill Problem Set 2, Due: Mar. 18, 2025}\\

(1) Use the probability integral transformation method to simulate from the distribution
\begin{gather}
    f(x) = 
    \begin{cases}
        \frac{2}{a^2}x,  & \text{if }0\leq x\leq a \\
        0, & \text{otherwise}
    \end{cases}
\end{gather}
where $a>0$. Set a value for $a$, simulate various sample sizes, and compare results to the true distribution.



\section*{Problem 1: Probability Integral Transformation}
Given the probability density function:
\begin{equation}
    f(x) = \begin{cases} \frac{2}{a^2} x, & 0 \leq x \leq a \\ 0, & \text{otherwise} \end{cases}
\end{equation}

To generate random samples using the probability integral transformation:

1. Start with a uniform random variable $U \sim U(0,1)$.
2. Compute the cumulative distribution function (CDF):
   \begin{equation}
       F(x) = \int_0^x \frac{2}{a^2} t dt = \frac{x^2}{a^2}, \quad 0 \leq x \leq a.
   \end{equation}
3. Solve for $X$ in terms of $U$: 
   \begin{equation}
       X = a\sqrt{U}.
   \end{equation}
4. Generate samples $X$ using $X = a\sqrt{U}$ for various values of $a$ and sample sizes.
5. Compare histograms of simulated data with the true density function.


\section*{Problem 2: Finite Mixture Approach}
The given density function is:
\begin{equation}
    f(x) = \frac{2}{3} e^{-2x} + 2 e^{-3x}.
\end{equation}
This represents a mixture of two exponential distributions:

1. With probability $\frac{1}{3}$, draw $X$ from $\text{Exp}(2)$.
2. With probability $\frac{2}{3}$, draw $X$ from $\text{Exp}(3)$.

Algorithm:
\begin{algorithm}
    \caption{Finite Mixture Sampling}
    \begin{algorithmic}[1]
        \For{$i = 1$ to $N$}
            \State Generate $U \sim U(0,1)$.
            \If{$U \leq \frac{1}{3}$}
                \State Draw $X_i \sim \text{Exp}(2)$.
            \Else
                \State Draw $X_i \sim \text{Exp}(3)$.
            \EndIf
        \EndFor
    \end{algorithmic}
\end{algorithm}

\newpage

(2) Generate samples from the distribution
\begin{gather}
    f(x)=\frac{2}{3}e^{-2x}+2e^{-3x}
\end{gather}
using the finite mixture approach.



\section*{Problem 3: Accept-Reject Algorithm for Beta(3,3)}
To sample from $\text{Beta}(3,3)$ using accept-reject:

1. Proposal distribution: Use $g(x) = 1$ for $x \sim U(0,1)$ (i.e., a uniform distribution).
2. Target density: $f(x) \propto x^2(1-x)^2$.
3. Compute the upper bound $M$ such that $M g(x) \geq f(x)$.
4. Generate candidate $Y \sim U(0,1)$ and accept with probability:
   \begin{equation}
       \frac{f(Y)}{M g(Y)}.
   \end{equation}
5. Repeat until 500 samples are drawn.
6. Compute sample mean and variance, comparing with theoretical values:
   \begin{equation}
       E[X] = \frac{3}{6} = 0.5, \quad \text{Var}(X) = \frac{3 \cdot 3}{(6)^2(7)} = \frac{9}{252} \approx 0.0357.
   \end{equation}

\end{document}

\newpage

(3) Draw 500 observations from Beta$(3,3)$ using the accept-reject algorithm. Compute the mean and variance of the sample and compare them to the true values.

\end{spacing}
\end{document}
